% Options for packages loaded elsewhere
\PassOptionsToPackage{unicode}{hyperref}
\PassOptionsToPackage{hyphens}{url}
%
\documentclass[
]{article}
\usepackage{lmodern}
\usepackage{amssymb,amsmath}
\usepackage{ifxetex,ifluatex}
\ifnum 0\ifxetex 1\fi\ifluatex 1\fi=0 % if pdftex
  \usepackage[T1]{fontenc}
  \usepackage[utf8]{inputenc}
  \usepackage{textcomp} % provide euro and other symbols
\else % if luatex or xetex
  \usepackage{unicode-math}
  \defaultfontfeatures{Scale=MatchLowercase}
  \defaultfontfeatures[\rmfamily]{Ligatures=TeX,Scale=1}
\fi
% Use upquote if available, for straight quotes in verbatim environments
\IfFileExists{upquote.sty}{\usepackage{upquote}}{}
\IfFileExists{microtype.sty}{% use microtype if available
  \usepackage[]{microtype}
  \UseMicrotypeSet[protrusion]{basicmath} % disable protrusion for tt fonts
}{}
\makeatletter
\@ifundefined{KOMAClassName}{% if non-KOMA class
  \IfFileExists{parskip.sty}{%
    \usepackage{parskip}
  }{% else
    \setlength{\parindent}{0pt}
    \setlength{\parskip}{6pt plus 2pt minus 1pt}}
}{% if KOMA class
  \KOMAoptions{parskip=half}}
\makeatother
\usepackage{xcolor}
\IfFileExists{xurl.sty}{\usepackage{xurl}}{} % add URL line breaks if available
\IfFileExists{bookmark.sty}{\usepackage{bookmark}}{\usepackage{hyperref}}
\hypersetup{
  pdftitle={Assigment 2 - ESM 232 Climate Modeling},
  pdfauthor={Anthony Luna, Chen Xing, Atefeh Mohseni},
  hidelinks,
  pdfcreator={LaTeX via pandoc}}
\urlstyle{same} % disable monospaced font for URLs
\usepackage[margin=1in]{geometry}
\usepackage{color}
\usepackage{fancyvrb}
\newcommand{\VerbBar}{|}
\newcommand{\VERB}{\Verb[commandchars=\\\{\}]}
\DefineVerbatimEnvironment{Highlighting}{Verbatim}{commandchars=\\\{\}}
% Add ',fontsize=\small' for more characters per line
\usepackage{framed}
\definecolor{shadecolor}{RGB}{248,248,248}
\newenvironment{Shaded}{\begin{snugshade}}{\end{snugshade}}
\newcommand{\AlertTok}[1]{\textcolor[rgb]{0.94,0.16,0.16}{#1}}
\newcommand{\AnnotationTok}[1]{\textcolor[rgb]{0.56,0.35,0.01}{\textbf{\textit{#1}}}}
\newcommand{\AttributeTok}[1]{\textcolor[rgb]{0.77,0.63,0.00}{#1}}
\newcommand{\BaseNTok}[1]{\textcolor[rgb]{0.00,0.00,0.81}{#1}}
\newcommand{\BuiltInTok}[1]{#1}
\newcommand{\CharTok}[1]{\textcolor[rgb]{0.31,0.60,0.02}{#1}}
\newcommand{\CommentTok}[1]{\textcolor[rgb]{0.56,0.35,0.01}{\textit{#1}}}
\newcommand{\CommentVarTok}[1]{\textcolor[rgb]{0.56,0.35,0.01}{\textbf{\textit{#1}}}}
\newcommand{\ConstantTok}[1]{\textcolor[rgb]{0.00,0.00,0.00}{#1}}
\newcommand{\ControlFlowTok}[1]{\textcolor[rgb]{0.13,0.29,0.53}{\textbf{#1}}}
\newcommand{\DataTypeTok}[1]{\textcolor[rgb]{0.13,0.29,0.53}{#1}}
\newcommand{\DecValTok}[1]{\textcolor[rgb]{0.00,0.00,0.81}{#1}}
\newcommand{\DocumentationTok}[1]{\textcolor[rgb]{0.56,0.35,0.01}{\textbf{\textit{#1}}}}
\newcommand{\ErrorTok}[1]{\textcolor[rgb]{0.64,0.00,0.00}{\textbf{#1}}}
\newcommand{\ExtensionTok}[1]{#1}
\newcommand{\FloatTok}[1]{\textcolor[rgb]{0.00,0.00,0.81}{#1}}
\newcommand{\FunctionTok}[1]{\textcolor[rgb]{0.00,0.00,0.00}{#1}}
\newcommand{\ImportTok}[1]{#1}
\newcommand{\InformationTok}[1]{\textcolor[rgb]{0.56,0.35,0.01}{\textbf{\textit{#1}}}}
\newcommand{\KeywordTok}[1]{\textcolor[rgb]{0.13,0.29,0.53}{\textbf{#1}}}
\newcommand{\NormalTok}[1]{#1}
\newcommand{\OperatorTok}[1]{\textcolor[rgb]{0.81,0.36,0.00}{\textbf{#1}}}
\newcommand{\OtherTok}[1]{\textcolor[rgb]{0.56,0.35,0.01}{#1}}
\newcommand{\PreprocessorTok}[1]{\textcolor[rgb]{0.56,0.35,0.01}{\textit{#1}}}
\newcommand{\RegionMarkerTok}[1]{#1}
\newcommand{\SpecialCharTok}[1]{\textcolor[rgb]{0.00,0.00,0.00}{#1}}
\newcommand{\SpecialStringTok}[1]{\textcolor[rgb]{0.31,0.60,0.02}{#1}}
\newcommand{\StringTok}[1]{\textcolor[rgb]{0.31,0.60,0.02}{#1}}
\newcommand{\VariableTok}[1]{\textcolor[rgb]{0.00,0.00,0.00}{#1}}
\newcommand{\VerbatimStringTok}[1]{\textcolor[rgb]{0.31,0.60,0.02}{#1}}
\newcommand{\WarningTok}[1]{\textcolor[rgb]{0.56,0.35,0.01}{\textbf{\textit{#1}}}}
\usepackage{graphicx,grffile}
\makeatletter
\def\maxwidth{\ifdim\Gin@nat@width>\linewidth\linewidth\else\Gin@nat@width\fi}
\def\maxheight{\ifdim\Gin@nat@height>\textheight\textheight\else\Gin@nat@height\fi}
\makeatother
% Scale images if necessary, so that they will not overflow the page
% margins by default, and it is still possible to overwrite the defaults
% using explicit options in \includegraphics[width, height, ...]{}
\setkeys{Gin}{width=\maxwidth,height=\maxheight,keepaspectratio}
% Set default figure placement to htbp
\makeatletter
\def\fps@figure{htbp}
\makeatother
\setlength{\emergencystretch}{3em} % prevent overfull lines
\providecommand{\tightlist}{%
  \setlength{\itemsep}{0pt}\setlength{\parskip}{0pt}}
\setcounter{secnumdepth}{-\maxdimen} % remove section numbering

\title{Assigment 2 - ESM 232 Climate Modeling}
\author{Anthony Luna, Chen Xing, Atefeh Mohseni}
\date{4/7/2021}

\begin{document}
\maketitle

\hypertarget{background}{%
\section{Background}\label{background}}

\hypertarget{almond-conceptual-model}{%
\subsection{Almond conceptual model}\label{almond-conceptual-model}}

The almond model derived from Lobell et al.~(2006) is a statistical
model calculating annual yield anomaly of almonds in California, which
is based on February minimum temperature (Celsius degree) and January
accumulated precipitation (mm).

\begin{Shaded}
\begin{Highlighting}[]
\KeywordTok{library}\NormalTok{(tidyverse)}
\KeywordTok{library}\NormalTok{(here)}
\end{Highlighting}
\end{Shaded}

\hypertarget{load-functions}{%
\subsection{Load function(s)}\label{load-functions}}

Load the almond model function!

The function will process:

\begin{itemize}
\tightlist
\item
  Turn the daily minimum temperature and daily precipitation into
  monthly results
\item
  Calculate annual almond yield conceptual model
\end{itemize}

\begin{Shaded}
\begin{Highlighting}[]
\KeywordTok{source}\NormalTok{(}\StringTok{"./R/almond_model.R"}\NormalTok{)}
\end{Highlighting}
\end{Shaded}

\hypertarget{read-input-data}{%
\subsection{Read input data}\label{read-input-data}}

Read the climate data in a txt format.

\begin{itemize}
\tightlist
\item
  There is a column containing time information
\item
  It includes two variables: minimum temperature and precipitation
\end{itemize}

\begin{Shaded}
\begin{Highlighting}[]
\NormalTok{clim_data_path <-}\StringTok{ }\KeywordTok{here}\NormalTok{(}\StringTok{"data"}\NormalTok{,}\StringTok{"clim.txt"}\NormalTok{)}
\end{Highlighting}
\end{Shaded}

\hypertarget{calculate-almond-yearly-yield-anomaly}{%
\subsection{Calculate almond yearly yield
anomaly}\label{calculate-almond-yearly-yield-anomaly}}

Use the defined function to calculate almond yield of each year in
California (units: ton/acre).

The almond yield equation (Lobell et al.~2006) is:

Y = -0.015* (Tn,2) - 0.0046* (Tn,2)\^{}2 - 0.07* (P1) + 0.0043*
(P1)\^{}2 + 0.28

where Y is almond yield anomaly (ton/acre); Tn,2 is February minimum
temperature (C); P1 is January precipitation (mm)

The yearly yield results will be returned to \textbf{yeild\_outcomes}

\begin{Shaded}
\begin{Highlighting}[]
\NormalTok{yeild_outcomes <-}\StringTok{ }\KeywordTok{almond_model}\NormalTok{(clim_data_path)}
\end{Highlighting}
\end{Shaded}

\hypertarget{results}{%
\section{Results}\label{results}}

\begin{Shaded}
\begin{Highlighting}[]
\NormalTok{yeild_plot <-}\StringTok{ }\KeywordTok{ggplot}\NormalTok{(}\DataTypeTok{data=}\NormalTok{yeild_outcomes)}\OperatorTok{+}
\StringTok{  }\KeywordTok{geom_line}\NormalTok{(}\KeywordTok{aes}\NormalTok{(}\DataTypeTok{x=}\NormalTok{year,}\DataTypeTok{y=}\NormalTok{yeild))}\OperatorTok{+}
\StringTok{  }\KeywordTok{labs}\NormalTok{(}\DataTypeTok{y =} \StringTok{"Total predicted yield"}\NormalTok{,}
    \DataTypeTok{x =} \StringTok{"Year"}\NormalTok{)}\OperatorTok{+}
\StringTok{    }\KeywordTok{theme_minimal}\NormalTok{() }\OperatorTok{+}\StringTok{  }
\StringTok{    }\CommentTok{# This centers our title and subtitle  }
\StringTok{    }\KeywordTok{theme}\NormalTok{(}\DataTypeTok{plot.title =} \KeywordTok{element_text}\NormalTok{(}\DataTypeTok{hjust=} \FloatTok{0.5}\NormalTok{),}
          \DataTypeTok{plot.subtitle =} \KeywordTok{element_text}\NormalTok{(}\DataTypeTok{hjust=} \FloatTok{0.5}\NormalTok{),}
        \DataTypeTok{axis.title.y =} \KeywordTok{element_text}\NormalTok{(}\DataTypeTok{angle =} \DecValTok{0}\NormalTok{,}\DataTypeTok{vjust =} \FloatTok{.5}\NormalTok{))}
\end{Highlighting}
\end{Shaded}

\begin{Shaded}
\begin{Highlighting}[]
\NormalTok{tmin_plot <-}\StringTok{ }\KeywordTok{ggplot}\NormalTok{(}\DataTypeTok{data=}\NormalTok{yeild_outcomes)}\OperatorTok{+}
\StringTok{  }\KeywordTok{geom_line}\NormalTok{(}\KeywordTok{aes}\NormalTok{(}\DataTypeTok{x=}\NormalTok{year,}\DataTypeTok{y=}\NormalTok{yeild_temp))}\OperatorTok{+}
\StringTok{  }\KeywordTok{labs}\NormalTok{(}\DataTypeTok{title=} \KeywordTok{str_wrap}\NormalTok{(}\StringTok{"Predicted almond yield based on regression model and climate anomoly contributions"}\NormalTok{,}\DecValTok{50}\NormalTok{),}
       \DataTypeTok{y =} \KeywordTok{str_wrap}\NormalTok{(}\StringTok{"Predicted yield due to February average minimum temperature"}\NormalTok{,}\DecValTok{20}\NormalTok{))}\OperatorTok{+}
\StringTok{  }\KeywordTok{theme_minimal}\NormalTok{()}\OperatorTok{+}
\StringTok{  }\KeywordTok{theme}\NormalTok{(}\DataTypeTok{axis.title.x =} \KeywordTok{element_blank}\NormalTok{(),}
        \DataTypeTok{axis.line.x =} \KeywordTok{element_blank}\NormalTok{(),}
        \DataTypeTok{axis.text.x=}\KeywordTok{element_blank}\NormalTok{(),}
        \DataTypeTok{axis.title.y =} \KeywordTok{element_text}\NormalTok{(}\DataTypeTok{angle =} \DecValTok{0}\NormalTok{,}\DataTypeTok{vjust =} \FloatTok{.5}\NormalTok{))}
\end{Highlighting}
\end{Shaded}

\begin{Shaded}
\begin{Highlighting}[]
\NormalTok{precip_plot <-}\StringTok{ }\KeywordTok{ggplot}\NormalTok{(}\DataTypeTok{data=}\NormalTok{yeild_outcomes)}\OperatorTok{+}
\StringTok{  }\KeywordTok{geom_line}\NormalTok{(}\KeywordTok{aes}\NormalTok{(}\DataTypeTok{x=}\NormalTok{year,}\DataTypeTok{y=}\NormalTok{yeild_precip))}\OperatorTok{+}
\StringTok{  }\KeywordTok{labs}\NormalTok{(}\DataTypeTok{y =} \KeywordTok{str_wrap}\NormalTok{(}\StringTok{"Predicted yield due to January total precipitation"}\NormalTok{,}\DecValTok{25}\NormalTok{))}\OperatorTok{+}
\StringTok{  }\KeywordTok{theme_minimal}\NormalTok{()}\OperatorTok{+}
\StringTok{  }\KeywordTok{theme}\NormalTok{(}\DataTypeTok{axis.title.x =} \KeywordTok{element_blank}\NormalTok{(),}
        \DataTypeTok{axis.line.x =} \KeywordTok{element_blank}\NormalTok{(),}
        \DataTypeTok{axis.text.x=}\KeywordTok{element_blank}\NormalTok{(),}
        \DataTypeTok{axis.title.y =} \KeywordTok{element_text}\NormalTok{(}\DataTypeTok{angle =} \DecValTok{0}\NormalTok{,}\DataTypeTok{vjust =} \FloatTok{.5}\NormalTok{))}
\end{Highlighting}
\end{Shaded}

\begin{Shaded}
\begin{Highlighting}[]
\KeywordTok{library}\NormalTok{(ggpubr)}

\NormalTok{ggpubr}\OperatorTok{::}\KeywordTok{ggarrange}\NormalTok{(tmin_plot,precip_plot,yeild_plot,}\DataTypeTok{nrow =} \DecValTok{3}\NormalTok{,}\DataTypeTok{align =} \StringTok{"v"}\NormalTok{)}
\end{Highlighting}
\end{Shaded}

\includegraphics{assignment-2_group-A2H_files/figure-latex/unnamed-chunk-4-1.pdf}

\hypertarget{discussion}{%
\section{Discussion}\label{discussion}}

A multiple linear regression model based on February minimum temperature
and January total precipitation was used to predict annual almond yield
anomalies and the contribution of each variable from 1988 to 2010. The
figures show that the February minimum temperature reduces almond yield
in the range of -0.4 to -0.9, with a relatively strong negative
contribution until 2000, but diminishes after 2000. However, for the
January precipitation, it produced positive apricot yield anomalies
ranging from 0 to 2,000 tons/acre. Considering the effects of each
variable and interceptions together, we obtain total annual almond yield
anomalies. Total yields were highly correlated with precipitation
predictions, with a similar magnitude and variation. Several peaks in
the total yield prediction are also identified in the precipitation
prediction, such as in 1995, 2005 and 2008. February minimum temperature
accounts for only 0.05\% of the magnitude of January precipitation.
Therefore, total apricot production was dominated by January
precipitation rather than the average minimum temperature in February.

\hypertarget{references}{%
\section{References}\label{references}}

Lobell D B, Field C B, Cahill K N, et al.~Impacts of future climate
change on California perennial crop yields: Model projections with
climate and crop uncertainties{[}J{]}. \emph{Agricultural and Forest
Meteorology}, 2006, \textbf{141(2-4)}: 208-218.

\hypertarget{appendix---almond_model.r}{%
\section{\texorpdfstring{Appendix -
\texttt{almond\_model.R}}{Appendix - almond\_model.R}}\label{appendix---almond_model.r}}

\begin{verbatim}
#' Almond Model
#'
#' This function computes the annual yield of almonds in California
#’ based on monthly minimum temperature in February and 
#' accumulated precipitation in January (Lobell et al. 2006)
#' @param clim_data file path to climate data that includes fields for 
#'   row_number, D, day, month, year, wy, tmax_C, tmin_c, precip, wyd
#' @param k1 coefficient for February minimum temperature changes 
#' @param k2 quadratic term coefficient for February minimum temperature changes
#' @param k3 coefficient for January Precipitation changes
#' @param k4 quadratic term coefficient for January Precipitation changes
#' @param intercept the intercept of the fitted line
#' @author Anthony Luna, Chen Xing, Atefeh Mohseni
#' @examples almond_model(["1" 1991-06-01 1 6 1991 1992 21.232 14.234 1.56 1])
#' @return Almonds yield anomaly (ton acre-1)
#' @references
#' Lobell D B, Field C B, Cahill K N, et al. Impacts of future climate change on 
#' California perennial crop yields: Model projections with climate and crop 
#' uncertainties[J]. Agricultural and Forest Meteorology, 2006, 141(2-4): 208-218.

almond_model <- function(clim_data, 
                         k1=-0.015 , 
                         k2=-0.0046 , 
                         k3=-0.07 ,
                         k4= 0.0043, 
                         intercept = 0.28 ) {
  ## Error checking
  ## You can use the library checkmate to
  ## make this easier, or just do it with ifelse statements
  
  # check clim_data dimensions
  no_col <- max(count.fields(clim_data, sep = " "))
  if (no_col < 10)
  {
    print("The input climate data file should have 10 columns")
  } else {
    # check clim_data column names
    file_header <- read.delim(clim_data, header=TRUE, sep = " ", dec = ".")
    file_headers = colnames(file_header)
    
    expected_header = c(
      "row_number",
      "D",
      "day",
      "month",
      "year",
      "wy",
      "tmax_c",
      "tmin_c",
      "precip",
      "wyd"
    ) 
    
    check_res <- isTRUE(all.equal(file_headers, expected_header))
    
    if (check_res) {
      # check clim_data column datatypes
      summary <- summary.default(file_header)
      numeric_types = length(summary[which(summary=="numeric")])
      date_types = length(summary[which(summary=="character")])
      
      if (numeric_types != 9 | date_types!= 1){
        print("There should have 9 numeric data type and 1 date type in the input file!")
      }
    }
    else{
      print("The format of the input file header is not as expected!")
    }
  }
  
  clim_data <- read_delim(file = clim_data, delim=" ")
  
  # check coefficients and intercept value
  
  ## End error checking
  
  # Turn the daily minimum temperature into monthly data through monthly average
  # and calculate monthly accumulated precipitation from daily station
  # precipitation data
  
  df_summarized <- clim_data %>% 
    select(year,month,day,tmin_c,precip) %>% 
    group_by(year,month) %>% 
    summarize(avg_tmin_c=mean(tmin_c),tot_precip = sum(precip)) %>% 
    ungroup()
  
  # Extract the February minimum temperature from the data summary
  
  df_temp <- df_summarized %>% 
    filter(month==2) %>% 
    select(-tot_precip,-month)
  
  # Extract the January precipitation from the data summary
  
  df_precip <- df_summarized %>% 
    filter(month==1) %>% 
    select(-avg_tmin_c,-month)
  
  # Calculate the almond annual yield from the statistical relationship:
  # Y = k1*(Tn,2) + k2*(Tn,2)^2 + k3*(P1) + k4*(P1)^2 + intercept
  # Y: almond yield anomaly (ton/acre); Tn,2: February minimum temperature (C); 
  # P1: January precipitation (mm)
  
  df_out <- full_join(df_temp,df_precip) %>% 
    mutate(yeild = k1*avg_tmin_c+k2*(avg_tmin_c^2)+k3*tot_precip+k4*(tot_precip^2)+intercept) %>% 
    mutate(yeild_temp = k1*avg_tmin_c+k2*(avg_tmin_c^2)) %>% 
    mutate(yeild_precip = k3*tot_precip+k4*(tot_precip^2)) %>% 
    mutate(intercept = intercept)
  
  # return the yield to the main function
  
  return(df_out)
}
\end{verbatim}

\end{document}
